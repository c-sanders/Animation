\begin{frame}[t]

	\frametitle{Definition of Euler's formula}

	So how did Euler derive this formula?\\~

	At the time he derived his formula, Euler was working with Infinite series representations of well known mathematical identities. From his work, he found that;
	\begin{equation} \label{eqn_Infinite_series_of_e^jtheta}
	e^{j\theta} = 1 + j\theta - \dfrac{\theta^2}{2!} - j\frac{\theta^3}{3!} + \frac{\theta^4}{4!} + j\frac{\theta^5}{5!} - \frac{\theta^6}{6!} - j\frac{\theta^7}{7!} + \cdots
	\end{equation}
	But Euler also knew that;
	\begin{align}
	cos(\theta) &= 1 - \dfrac{\theta^2}{2!} + \frac{\theta^4}{4!} - \frac{\theta^6}{6!} + \cdots      \label{eqn_slide_3_a} \\
	sin(\theta) &= \theta - \frac{\theta^3}{3!} + \frac{\theta^5}{5!} - \frac{\theta^7}{7!} + \cdots  \label{eqn_slide_3_b}
	\end{align}

\end{frame}


\begin{frame}[t]

	\frametitle{Definition of Euler's formula ... (continued)}

	Re-arranging the Infinite series representation of \(e^{j\theta}\) from the previous slide, so that all \(j\) terms are collected together yields;
	\begin{equation} \label{eqn_slide_4_a}
	e^{j\theta} = \Big(1 - \dfrac{\theta^2}{2!} + \frac{\theta^4}{4!} - \frac{\theta^6}{6!} + \cdots\Big) + j\Big(\theta - \frac{\theta^3}{3!} + \frac{\theta^5}{5!} - \frac{\theta^7}{7!} + \cdots\Big)
	\end{equation}

	From this rearranged version of the equation, it is now obvious how Euler came up with his formula. That is, the first term within

\end{frame}
