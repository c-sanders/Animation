\documentclass[aspectratio=169, final]{beamer}

\usepackage[utf8]{inputenc}
\usetheme{Copenhagen}
\usecolortheme{orchid}
% \usefonttheme{serif}
\usefonttheme[onlymath]{serif}

\usepackage{amsmath, amsthm, amssymb, amsfonts}
\usepackage{mathtools}
\usepackage{commath}
\usepackage{graphicx}

% \usepackage[active]{preview}

\title{3D fly-around of Euler's formula.}
\author{Craig Sanders}
\institute{Gravitas Toolworks}
\date{2018}

\begin{document}

% --------------------
% Frame : Title page
% --------------------

\frame{\titlepage}

% --------------------
% Frame : TOC
% --------------------

\iffalse

\begin{frame}[t]

	\frametitle{Table of Contents}

	\tableofcontents

\end{frame}

\fi


% --------------------
% Frame : 0
% --------------------

\begin{frame}[t]

	\frametitle{Definition of Euler's number}

	Recall that Euler's number can be defined as follows;

	\begin{flalign} \label{eqn_frame_1_a}
	e &= \sum_{x=0}^{\infty}\frac{1}{x!} \\
	  &= \frac{1}{0!} + \frac{1}{1!} + \frac{1}{2!} + \frac{1}{3!} + ... \\
	  &= \frac{1}{1} + \frac{1}{1} + \frac{1}{2} + \frac{1}{6} + ... \\
	  &= 1 + 1 + \frac{1}{2} + \frac{1}{6} + ...
	\end{flalign}

\end{frame}

% --------------------
% Frame : 1
% --------------------

\begin{frame}[t]

	\frametitle{Definition of \(sin(x)\) and \(cos(x)\)}

	\begin{flalign} \label{eqn_frame_1_a}
	sin(x) &= \sum_{n=0}^{\infty}\frac{(-1)^n x^{2n+1}}{(2n+1)!} \\
		   &= \frac{x^1}{1!} - \frac{x^3}{3!} + \frac{x^5}{5!} - \frac{x^7}{7!} + \cdots
	\end{flalign}

	\begin{flalign} \label{eqn_frame_1_a}
	cos(x) &= \sum_{n=0}^{\infty}\frac{(-1)^n x^{2n}}{(2n)!} \\
		   &= \frac{x^0}{0!} - \frac{x^2}{2!} + \frac{x^4}{4!} - \frac{x^6}{6!} + \cdots
	\end{flalign}

\end{frame}

% --------------------
% Frame : 1
% --------------------

\begin{frame}[t]

	\frametitle{Definition of Euler's formula}

	Recall that Euler's formula is defined as follows;

	\begin{equation} \label{eqn_frame_1_a}
	e^{j\theta} = cos(\theta) + jsin(\theta)
	\end{equation}

	This appears easy enough to comprehend, doesn't it?\\

	However, let's not be quite so quick.\\

	What does it actually mean to raise a number -- \(e\) in this case, to the power of a complex number, \(j\theta\) in this case?

\end{frame}


% --------------------
% Frame : 2
% --------------------

\begin{frame}[t]

	\frametitle{Expanding Euler's formula}

	\begin{equation} \label{eqn_frame_2_a}
	e^{j\theta} = cos(\theta) + jsin(\theta)
	\end{equation}

	Let \(a = e\) and replace the purely Imaginary \(j\theta\) with the Complex (and thus more general), \(\sigma + j\omega\). Doing so gives;

	\begin{flalign} \label{eqn_frame_2_b}
	e^{j\theta} &= a^{\sigma + j\omega}
	\end{flalign}

	Recall that \(a = e^{ln(a)}\), thus;

	\begin{flalign} \label{eqn_frame_2_c}
	e^{j\theta} &= (e^{ln(a)})^{\sigma + j\omega} \\
	e^{j\theta} &= e^{ln(a)\sigma + ln(a)j\omega} \\
	e^{j\theta} &= e^{ln(a)\sigma}e^{j(ln(a)\omega)}
	\end{flalign}

\end{frame}


% --------------------
% Frame : 2a
% --------------------

\begin{frame}[t]

	\frametitle{Expanding Euler's formula (continued)}

	\begin{flalign} \label{eqn_frame_2_d}
	e^{j\theta} &= e^{ln(a)\sigma}\big[cos\big(ln(a)\omega\big) + jsin\big(ln(a)\omega\big)\big]
	\end{flalign}

\end{frame}


% --------------------
% Frame : 3
% --------------------

\begin{frame}[t]

	\frametitle{Raising \(e\) to the power of a Real number}

	Raising \(e\) to the power of a Real number is rather straightforward.\\

	\(e\) raised to the power of 0, 1, and 2, is defined as follows.

	\begin{flalign} \label{eqn_frame_3_a}
	e^{0} &= 1   \\
	e^{1} &= 1 \times e   \\
	e^{2} &= 1 \times e \times e
	\end{flalign}

\end{frame}


% --------------------
% Frame : 3
% --------------------

\begin{frame}[t]

	\frametitle{Raising \(e\) to the power of an Imaginary number}

	However, what happens if we want to raise \(e\) to the power of a purely Imaginary (as opposed to a Complex) number?

	\begin{equation} \label{eqn_frame_3_a}
	e^{j\theta} = ?
	\end{equation}

	When \(\theta = 0\) we get;

	\begin{equation} \label{eqn_frame_3_b}
	e^{0} = 1
	\end{equation}

	and when \(\theta = \pi/2\) we get;

	\begin{equation} \label{eqn_frame_3_c}
	e^{j\pi/2} = j
	\end{equation}

	Similarly, when \(\theta = \pi, 3\pi/2\), and \(2\pi\) we get;

\end{frame}


\end{document}
