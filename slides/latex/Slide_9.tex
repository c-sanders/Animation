\begin{frame}[t]

	\frametitle{Raising a number to an Imaginary power}

	Now consider what happens if we raise a number to an Imaginary power.\\~

	As an example, consider the following mathematical expression;
	\begin{equation} \label{eqn_slide_9_a}
	e^{jn}
	\end{equation}

	What does this actually mean?\\~

	It can't mean that \(e\) gets multiplied by itself \(n\) times, as this is
	what happens when the number is raised to a Real power!\\~

	Instead, what it means is that \(e\) gets multiplied by \(j\), \(n\) times.

	% In the previous slide, we saw that the result of raising a number to a Real power, was a value on the Real number line.\\~

	% However, in this case, the result cannot possibly reside on the Real number line, can it? Afterall, we are now raising the number to an Imaginary power.

\end{frame}
