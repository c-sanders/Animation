\begin{frame}[t]

	\frametitle{Definition of Euler's formula}

	The equality which Euler discovered is expressed by way of the following equation;

	\begin{equation} \label{eqn_frame_1_a}
	e^{j\theta} = cos(\theta) + jsin(\theta)
	\end{equation}

	This equation -- in all its beautiful simplicity, is Euler's formula!\\~

	Note that the right-hand side of the formula is comprised of two components, they being \(cos(\theta)\) and \(jsin(\theta)\).

	\begin{itemize}
		\item   The first component \(cos(\theta)\), will only yield results which are purely \textbf{Real} in value.
		\item   The second component \(jsin(\theta)\), will only yield results which are purely \textbf{Imaginary} in value.
		\item   The outcome of adding these two components together \(e^{j\theta}\), will therefore yield results which are \textbf{Complex} in value.
	\end{itemize}

\end{frame}
